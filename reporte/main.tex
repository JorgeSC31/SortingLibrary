\documentclass[letterpaper, 12pt]{article}
\usepackage[utf8]{inputenc}
\usepackage{enumerate}
\usepackage{amsmath}
\usepackage{amssymb}
\usepackage{parskip}
\usepackage{physics}
\usepackage{xcolor}
\usepackage[spanish, es-nodecimaldot]{babel}
\usepackage[
    letterpaper,
    top = 25mm,
    bottom = 25mm,
    left = 25mm,
    right = 25mm
]{geometry}


\definecolor{darkgray}{rgb}{0.66, 0.66, 0.66}

\title{Librería de algoritmos de ordenamiento \\ 
Estructuras de Datos y Algoritmos}
\author{Jorge Salazar Cruz \\
    jorge.salazar@cimat.mx\\ \\
  \multicolumn{1}{p{.7\textwidth}}{\centering\emph{
  Universidad de Guanajuato}}}
\date{Junio 2020}

\begin{document}


\maketitle

\section{Introducción}
En ciencias de la computación, un \textbf{algoritmo de ordenamiento} es un algoritmo que reordena los elementos de una lista o un vector de acuerdo a una relación de orden. La relación de orden es usada para decidir el nuevo orden de los elementos en la respectiva estructura de datos. 

Mas formalmente, la salida de un algoritmo de ordenamiento debe satisfacer las siguientes dos condiciones:
\begin{enumerate}
    \item La salida está en un orden no decreciente.
    \item La salida es una permutación de la entrada.
\end{enumerate}

Los algoritmos de ordenamiento usualmente se clasifican por:

\begin{itemize}
    \item \textbf{Complejidad computacional} (Peor caso, promedio y mejor caso) en términos del tamaño de la entrada $N$. Los mejores algoritmos de ordenamiento tienen complejidad de $O(N\ log_2 N)$ y los que tienen peor complejidad tiene $O(N^2)$.
    
    \item \textbf{Memoria} usada (y uso de otros recursos de la computadora). 
    
    \item \textbf{Estabilidad} los algoritmos de ordenamiento estables mantiene el orden relativo que tenían originalmente los elementos con claves iguales.
\end{itemize}




\section{Librería de ordenamiento}
Se desarrollo la librería \colorbox{darkgray}{sorting\_library.hpp} en C++. Está esta librería están implementados 6 de los mas populares algoritmos de ordenamoento: \textbf{Buuble Sort, Insertion Sort, Selection Sort, Merge Sort, Quick Sort, Heap Sort}. 



%Tabla de comparación
%Algoritmo - complejidad - memoria - estabilidad - peor caso - caso promedio - mejor caso
%Prueba 
%Algoritmo - entrada - tiempo

\end{document}
